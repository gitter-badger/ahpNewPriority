\documentclass{article}
\usepackage{amsfonts}
\usepackage{amssymb}
\usepackage{amsmath}
\usepackage{amsthm}
\usepackage{soul}
\newtheorem{definition}{Definition}
\title{New Eigenvector-type Calculation}
\author{Bill Adams}
\begin{document}
\maketitle
\begin{abstract}
These are some notes on my ideas after seeing the unusual behavior
of the eigenvector calculation on some pairwise comparison matrices.
Essentially, when one inverts all of the votes in pairwise comparison
matrix, the resulting priority vector may not be inverted.  In fact,
even the ordering may not be reversed.
\end{abstract}

\section{Introduction}
Consider the following matrix:
$$A =\begin{bmatrix}
1   & 1/3 &  3  & 1   \\
3   & 1   & 1/3 & 3   \\
1   & 3   & 1   & 1/9 \\
1   & 1/3 & 9   & 1
\end{bmatrix}
$$
The matrix of the inverse votes is the same as the transpose $A^T$:
$$A^T =\begin{bmatrix}
1   & 3   & 1/3 & 1   \\
1/3 & 1   & 3   & 1/3 \\
1   & 1/3 & 1   & 9   \\
1   & 3   & 1/9 & 1
\end{bmatrix}
$$
If we let $\lambda$ be the function that returns the largest
eigenvector of a matrix, we have the following from the standard
calculations.
\begin{eqnarray*}
	\lambda(A)&=&(0.1788,0.2967, 0.1797, 0.3448) \\
	\lambda(A^T)&=&(0.1868, 0.2401, 0.4000, 0.1731)
\end{eqnarray*}
Notice that the rankings are:
\begin{eqnarray*}
	\mathrm{Rank}(A)&=&(4, 2, 3, 1) \\
	\mathrm{Rank}(A^T)&=&(3, 2, 1, 4)
\end{eqnarray*}
In other words, the rankings do not reverse as expected!

\section{Calculation}
The new calculation is based upon taking the geometric average of
the columns, a technique already described elsewhere.  The first
basic calculation is simply that, but let's define it anyways.

\begin{definition}[Geometric Average]
\label{def:geomavg1}
	Let $\mathbf{s}=(s_1, \ldots, s_n)$ be a sequence of non-negative
	numbers.  The \ul{geometric average} of $\mathbf{s}$, denoted by
	$\widehat{\mu}(\mathbf{s})$ is given by:
	$$\widehat{\mu}(\mathbf{s}) =
	\left(\prod_{s_i \neq 0}s_i\right)^{1/m}$$
	where $m$ is the number of $s_i$'s not equal to zero, if $m\neq 0$.
	and $1$ otherwise.
\end{definition}

\begin{definition}[Geometric average priority calculation]
	If $A$ is a pairwise comparison matrix of size $n$ we define the geometric
	average of the columns $\mathbf{g}(A)$ as the column vector
	defined by the formula:
	$$\mathbf{g}(A)_{i} = \widehat{\mu}(A_{,i})$$
	where $A_{,i}$ is the $i^{th}$ column of $A$.
\end{definition}

If we come up with an alternate definition of matrix multiplication
we can express the previous definition differently.

\begin{definition}[BE multiplication $\circledast$ for vectors]
	If $V$ is a row vector and $W$ is a column vector, both of size
	$n$, then we define their \ul{BE product} by:
	$$V \circledast W := \widehat{\mu}(V \bigodot W)$$
	where $\bigodot$ represents the Hadamard product, i.e.
	component-wise multiplication.
\end{definition}

\begin{definition}[BE multiplication $\circledast$ for matrices]
	If $M$ is an $(m,n)$ matrix and $N$ is an $(n,q)$ matrix, their
	BE product is an $(m,q)$ matrix, whose entries are given by:
	$$
	\left(M \circledast N\right)_{i,j} =
	A_{i,} \circledast A_{, j}
	$$
	where $A_{i,}$ is the $i^{th}$ row of $A$ and $A_{,j}$ is the
	$j^{th}$ column of $A$.
\end{definition}

With this notation, and if we write $E_n$ for the column vector of
size $n$ all of whose entries are 1, we can rewrite Definition
\ref{def:geomavg1}.

\begin{definition}[Geometric average priority calculation V2]
	If $M$ is an $(n,n)$ pairwise comparison matrix, we can rewrite
	the geometric average calculation as follows:
	$$
	\mathbf{g}(A) = M \circledast E_n
	$$
\end{definition}

\begin{definition}[Bill's Priority Calculation]
	If $M$ is an $(n,n)$ pairwise comparison matrix, we define the
	\ul{Bill Priority Calculation} in terms of the following sequence.
	\begin{eqnarray*}
		v_1 &=& M \circledast E_n \\
		v_{k+1} &=& M \circledast v_k
	\end{eqnarray*}
	And we define the Bill Priority of the pairwise comparison
	matrix $M$ as:
	$$\mathbf{\beta}(M) = \lim_{k \to \infty} v_k$$
\end{definition}

\end{document}
